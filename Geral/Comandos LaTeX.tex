% PACKAGES ======================================
\documentclass{article}
\usepackage[brazil]{babel}
\usepackage[utf8]{inputenc}
\usepackage{amsmath}
\usepackage{amsthm}
\usepackage{amssymb}
\usepackage{natbib}
\usepackage{graphicx}
\usepackage{shortvrb}


% TITLE ==========================================
\title{\textbf{Primeiro Projeto}}
\author{Vítor Martin Simoni}
%\date{20/02/2020}

\begin{document}

\maketitle

\pagebreak


% SUMMARY TABLES =================================
\tableofcontents
\listoffigures
\listoftables

\pagebreak


% SECTION =========================================
This is the text for outside any seciton.

\section{Section}
Para usar SECTION sem numeração, usar comando \verb \section* .

\subsection{Subsection}
This is the text for the sub-section.

\subsubsection{Sub-sub-section}
This is the text for the sub-sub-section.

\begin{figure}[h!]
    \centering
    \includegraphics[scale=.5]{fig1}
    \caption{Esta é minha primeira figura.}
    \label{fig:my_label}
\end{figure}


% VARIAVEIS =======================================
\section{Declarando Variáveis}

Para declarar novas variáveis, utilize o seguinte comando:

\begin{verbatim}
    \newcommand\name_of_variable{value_of_variable}
\end{verbatim}

\noindent
Para utilizar a variável, utilize o seguinte comando:

\begin{verbatim}
    \name_of_variable
\end{verbatim}

\newcommand\nome{banana}
\newcommand\valor{5}

\pagebreak
\noindent
Testando novas variáveis:
\begin{quote}


\begin{verbatim}
Uma var possui o texto ``\nome'' e outra, o valor \valor .
\end{verbatim}
\smallskip

    Uma var possui o texto ``\nome'' e outra, o valor \valor .
\end{quote}

\noindent
Utilize \verb \verb|caracter|  para desconsiderar a função de caracteres especiais, como \verb \ , \verb|{}| e até \verb|e s  p   a    ç     o      s|.

\medskip
\noindent
Para ainda mais detalhes, utilize \verb|\verb*| (com asterisco) para mostrar, inclusive, \verb*|e s p a ç o s|.

% OPERADORES LOGICOS ==============================
\section{Operadores Lógicos}
Use o cifrão para indicar que é uma equação:
\medskip % \smallskip ou \medskip ou \bigskip

\begin{tabular}[h!]{l}
    $A \wedge B$\\
    $A \vee B$\\
    $\neg A$\\
    $A \rightarrow B$\\
    $A \leftrightarrow B$\\
\end{tabular}

\subsection{Algumas Fórmulas Lógicas}
Use \textit{\textbf{align*}} (com asterisco) para não mostrar a numeração das linhas automaticamente.
%
\begin{align*}
    1. \hspace{10pt}  &   E \rightarrow Q         &   \text{(hip.)}                             \\
    2. \hspace{10pt}  &   E \vee B                &   \text{(hip.)}                           \\
    3. \hspace{10pt}  &   \neg Q                  &   \text{(hip.)}                           \\
    4. \hspace{10pt}  &   \neg E                  &   \text{(1, 3 } \rightarrow \text{ mt)}   \\
    5. \hspace{10pt}  &   \neg E \rightarrow B    &   \text{(2 } \rightarrow \text{ imp)}     \\
    6. \hspace{10pt}  &   B \qed                  &   \text{(4, 5 } \rightarrow \text{ mp)}   
\end{align*}


% TABELA ==========================================
\section{Tabela}

Para bold em equações (\verb|$$|), utilizar comando \verb|\mathbf{}|

\verb|$\mathbf{x + y + z = 0}$|

$\mathbf{x + y + z = 0}$

\begin{table}[h!]
    \centering
    \begin{tabular}{||r l||}
        \hline
        \textbf{Coluna} 1 & \textbf{Coluna 2} \\
        \hline
        \hline
        linha 1 & abc \\
        \hline
        linha 2 & def \\
        \hline
    \end{tabular}
    \caption{Comandos do Latex}
    \label{tab:my_label}
\end{table}




















% BIBLIOGRAFIA ====================================
\bibliographystyle{plain}
\bibliography{References}


\end{document}
