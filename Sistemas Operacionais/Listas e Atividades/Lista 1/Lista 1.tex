\documentclass{article}
\usepackage[brazil]{babel}
\usepackage[utf8]{inputenc}
\usepackage{graphicx}
\usepackage{natbib}
\usepackage{amsmath}

\title{\textbf{
    SISTEMAS OPERACIONAIS \\
    Lista 1
    }
}
\author{
\begin{tabular}{l r}
    Bruna Galastri Guedes & 18.00189-0 \\
    Daniel Ughini Xavier  & 18.00022-3 \\
    Leonardo Cury Haddad  & 18.00442-3 \\
    Rodolfo Cochi         & 18.00202-0 \\
    Vítor Martin Simoni   & 18.00050-9
\end{tabular}
}
\date{02/03/2020}

\begin{document}
\maketitle
\pagebreak

% 1 ===========================================================
\section*{1. \large Quais são as duas principais funções de um sistema operacional?}

\medskip
\noindent
\textbf{R:} Controlar o Hardware e permitir a execução de softwares.

% 2 ===========================================================
\section*{2. \large Qual é a diferença entre os sistemas timesharing e multiprogramação?}

\medskip
\noindent
\textbf{R:} O timesharing, possibilita que vários programas sejam executados através de uma divisão de tempo do processador em intervalos curtos enquanto multiprogramação é uma evolução dos programas de monoprogramação. Possibilitando, por exemplo, que enquanto um programa esteja em operação de leitura outros possam ser executados.

% 3 ===========================================================
\section*{3. \large Nos primeiros computadores, cada byte de dados lidos ou escritos era manipulado pela CPU (ou seja, não havia DMA – acesso direto à memória, que dispensa a CPU de ficar manipulando os dados de/para memória). Que implicações isso tem para a multiprogramação?}

\medskip
\noindent
\textbf{R:} Já que com a multiprogramação você tem vários processos sendo executados sequencialmente com alternância rápida e constante, se todas operações de dados envolverem E/S, sempre que pelo menos um dos processos estiver executando E/S, todos os outros tëm que esperar pelo término da operação. Isto pode trazer perda de desempenho muito grande ao sistema.

% 4 ===========================================================
\section*{4. \large As instruções relacionadas ao acesso a dispositivos de E/S são geralmente instruções privilegiadas, isto é, elas podem ser executadas no modo kernel, mas não no modo de usuário. Dê uma razão pela qual essas instruções são privilegiadas.}

\medskip
\noindent
\textbf{R:} Resposta da 4.

% 5 ===========================================================
\section*{5. \large Qual é a diferença entre o modo kernel e o modo de usuário? Explique como ter dois distintos modos ajuda na criação de um sistema operacional.}

\medskip
\noindent
\textbf{R:} A principal diferença entre o modo usuario para o modo kernel é que neste se tem acesso irrestrito ao conjunto de instruções da máquina.

% 6 ===========================================================
\section*{6. \large Uma CPU tem um pipeline com quatro etapas. Cada estágio leva o mesmo tempo para fazer o trabalho e que é 1ns. Quantas instruções por segundo esta máquina pode executar?}

\medskip
\noindent
\textbf{R:}Tendo em mente que a CPU tem um pipeline com quatro etapas e que cada estágio leva 1ns para ser concluído, temos que cada instrução leva 4ns para ser executada, portanto em um intervalo de 1s poderão ser executadas um total de $2,5.10^8$ instruções.

% 7 ===========================================================
\section*{7. \large As máquinas virtuais tornaram-se muito populares por vários motivos. Mesmo assim, elas têm algumas desvantagens. Nomeie uma.}

\medskip
\noindent
\textbf{R:} Uma das desvantagens do uso de máquinas virtuais é a sobregarga de tarefas que o uso exagerado das mesmas pode causar, levando a um mal funcionamneto ou até mesmo a falha de todas. Outra desvantagem relacionada ao uso de máquinas virtuais, é que caso tenha problema de vulnerabilidade em apenas uma, as outras podem ser comprometidas também.

% 8 ===========================================================
\section*{8. \large Em todos os computadores atuais, pelo menos parte dos tratadores de interrupção são escritos em linguagem assembly. Por quê?}

\medskip
\noindent
\textbf{R:} Parte dos tratadores de interrupção são escritos em linguagem de máquina porque ações como salvar os registradores e alterar o ponteiro de pilha não podem ser expressas em linguagens de alto nível, assim elas são implementadas por uma pequena rotina em linguagem assembly.

% 9 ===========================================================
\section*{9. \large Quando uma interrupção ou uma chamada do sistema transfere o controle para o sistema operacional, um uma área de pilha do kernel, separada do processo que foi interrompido, é geralmente usada. Por quê?}

\medskip
\noindent
\textbf{R: } Resposta da 9.

\end{document}